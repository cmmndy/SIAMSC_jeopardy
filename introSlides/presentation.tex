\documentclass{beamer}
\usepackage{float}
\usepackage{graphicx}


\begin{document}
	
	\begin{frame}
		\begin{center}
		{\includegraphics[scale = 0.3]{greeting.png}}
		\end{center}
	\end{frame}
	
	\begin{frame}
		\frametitle{Science what?}
		Jeopardy is
		\begin{itemize}
			\item a famous quiz show where the participants have to guess the correct \textbf{question}.
			\item a game, played by three to five players at a time.
			\item your chance to enjoy free drinks and win awesome prizes!
			\item fun!
		\end{itemize}
	\end{frame}
	
	\begin{frame}
		\frametitle{Credit}
		Food and drinks are sponsored by 
			\begin{center}
				{\includegraphics[scale = 0.3]{../orga/d-fine_cmyk_100_75_10_60_pz.jpg}}
			\end{center}
			\pause
		Awesome prizes sponsored by 
		\begin{center}
			{\includegraphics[scale = 0.3]{../orga/d-fine_cmyk_100_75_10_60_pz.jpg}} 
			\newline
			{\includegraphics[scale = 0.3]{mathworks.jpg}}
		\end{center}
	\end{frame}
		
	\begin{frame}
		\frametitle{Rules I}
		\begin{itemize}
			\item 3 to 5 players play at a time
			\item Questions in different categories, with difficulty 100 to 500 
			\item At the beginning of a round, I will explain the categories and what we are looking for.
			\item The winner of the last questions gets to select the topic and difficulty of the next question
			\item The person that hits the buzzer first gets the right to answer
		\end{itemize}
	\end{frame}
	
		\begin{frame}
			\frametitle{Rules II}
			\begin{itemize}
			\item Quick: Buzzertest!
			\item Correct answer: Score +X00 points. 
			\item Wrong answer: Score -X00 points.
			\item Person with the most points at the end of the game enters the final round (also: Lucky Loser).
			\item Bonus Field: Jeopardy Double (exclusive to the winner of the last question).
		\end{itemize}
	\end{frame}
		

	
	\begin{frame}
		\frametitle{Example I }
		You choose a category: "Famous landmarks for 200, please."
		\vspace{1cm}
		\pause
		
		A: "This famous landmark is placed in the so called city of love."
		
		\vspace{1cm}
		
		\pause 
		Q: "What is the Eiffel Tower." {\color{green}{Right! +200}}
		
		\pause
		A: "The Eiffel Tower."  {\color{red}{Wrong! -200}}
		
		
		
		
		
	\end{frame}	
	
		\begin{frame}
			\frametitle{Example II }
			The winner of the last question choses: "History of science for 500, please."
			\vspace{1cm}
			\pause
			
			A: "This person gave his name for the yearly awarded, prestigious scientific prize. Apparently, he didn't like mathematics..."
			
			\vspace{1cm}
			
			\pause
			A: "The Nobel Prize." \pause {\color{red}{Wrong! -500}}
			
			\pause
			Q: "What is the Nobel Prize." \pause {\color{red}{Wrong! -500}}
						
			\pause
			Q: "Who was Alfred Nobel." {\color{green}{Right! +500}}
			
			
			
		\end{frame}	
	
	
	
	\begin{frame}
		\frametitle{Who wants to play?}
		\begin{itemize}
		\item 3 Qualification rounds, with 4 to 5 players each.
		\item 1 Final round, with the 3 winners and 1 to 2 Lucky Loser (highest score).
		\end{itemize}
	\end{frame}
	
	\begin{frame}
		\begin{center}
		Questions?
		
		No? 
		
		Good, because you will be the jury!
		\end{center}
	\end{frame}

	
	
	
	
	
\end{document}